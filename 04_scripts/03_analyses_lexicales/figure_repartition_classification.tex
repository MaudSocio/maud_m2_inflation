\documentclass{article}
\usepackage{array}
\usepackage{booktabs}
\usepackage{longtable}
\usepackage{caption}
\usepackage{geometry}

\geometry{left=0.2cm, right=0.2cm, top=0.2cm, bottom=0.2cm}

\newlength\holdLTleft
\newlength\holdLTright
\setlength\holdLTleft{\LTleft}\relax
\setlength\holdLTright{\LTright}\relax
\setlength\LTleft{0\linewidth}
\setlength\LTright{0\linewidth}
\setlength{\LTpost}{0mm}

\begin{document}

\begin{longtable}{@{\extracolsep{\fill}}>{\raggedright\arraybackslash}p{\dimexpr 0.05\linewidth-2\tabcolsep-1.5\arrayrulewidth}|p{\dimexpr 0.09\linewidth-2\tabcolsep-1.5\arrayrulewidth}p{\dimexpr 0.09\linewidth-2\tabcolsep-1.5\arrayrulewidth}p{\dimexpr 0.09\linewidth-2\tabcolsep-1.5\arrayrulewidth}p{\dimexpr 0.09\linewidth-2\tabcolsep-1.5\arrayrulewidth}p{\dimexpr 0.09\linewidth-2\tabcolsep-1.5\arrayrulewidth}p{\dimexpr 0.09\linewidth-2\tabcolsep-1.5\arrayrulewidth}p{\dimexpr 0.09\linewidth-2\tabcolsep-1.5\arrayrulewidth}p{\dimexpr 0.09\linewidth-2\tabcolsep-1.5\arrayrulewidth}p{\dimexpr 0.09\linewidth-2\tabcolsep-1.5\arrayrulewidth}p{\dimexpr 0.09\linewidth-2\tabcolsep-1.5\arrayrulewidth}p{\dimexpr 0.09\linewidth-2\tabcolsep-1.5\arrayrulewidth}}


\caption*{
{\large \textbf{          Figure 18. Répartition des paragraphes publiés par semestre selon la classe affectée par la classification Reinert}}
} \\ 
\toprule
\multicolumn{1}{l}{} & \textbf{1. Pouv. achat} & \textbf{2. Prod. Ent.} & \textbf{3. Valo. Sal.} & \textbf{4. Débats pol.} & \textbf{5.
Ville} & \textbf{6. Insee} & \textbf{7. Ukr. Energie} & \textbf{8.March. Finan.} & \textbf{9. Dette. Pol.moné.} & \textbf{10.BCE} & \textbf{Total} \\ 
\midrule\addlinespace[2.5pt]
\endfirsthead
\caption*{(\textit{suite})} \\
\toprule
\multicolumn{1}{l}{} & \textbf{acheter-petit} & \textbf{produire-marque} & \textbf{salarié-prime} & \textbf{député-ministre} & \textbf{municipal-ville} & \textbf{insee-atteindre} & \textbf{ukraine-pétrole} & \textbf{nasdaq-dow} & \textbf{dette-état} & \textbf{taux-bce} & \textbf{Total} \\ 
\midrule\addlinespace[2.5pt]
\endhead
\midrule
\multicolumn{12}{r}{\textit{Suite à la page suivante}}
\endfoot
\bottomrule
\endlastfoot
S1 2020 & $9,1 \%$ & $13,5 \%$ & $15,0 \%$ & $6,5 \%$ & $12,3 \%$ & $8,2 \%$ & $1,8 \%$ & $0,0 \%$ & $30,5 \%$ & $3,2 \%$ & $100 \%$ \\ 
S2 2020 & $11,3 \%$ & $5,0 \%$ & $2,5 \%$ & $5,0 \%$ & $4,2 \%$ & $16,7 \%$ & $3,3 \%$ & $0,0 \%$ & $27,2 \%$ & $24,7 \%$ & $100 \%$ \\ 
S1 2021 & $5,8 \%$ & $5,2 \%$ & $2,8 \%$ & $2,4 \%$ & $1,5 \%$ & $14,2 \%$ & $5,3 \%$ & $1,5 \%$ & $39,5 \%$ & $21,9 \%$ & $100 \%$ \\ 
S2 2021 & $3,5 \%$ & $8,2 \%$ & $18,0 \%$ & $6,2 \%$ & $3,3 \%$ & $15,1 \%$ & $6,0 \%$ & $0,3 \%$ & $26,3 \%$ & $13,0 \%$ & $100 \%$ \\ 
S1 2022 & $7,7 \%$ & $14,8 \%$ & $11,7 \%$ & $7,1 \%$ & $4,8 \%$ & $13,3 \%$ & $7,8 \%$ & $0,2 \%$ & $21,6 \%$ & $10,9 \%$ & $100 \%$ \\ 
S2 2022 & $15,3 \%$ & $14,3 \%$ & $14,4 \%$ & $9,1 \%$ & $10,2 \%$ & $10,1 \%$ & $3,1 \%$ & $0,1 \%$ & $15,0 \%$ & $8,3 \%$ & $100 \%$ \\ 
S1 2023 & $16,2 \%$ & $21,9 \%$ & $8,9 \%$ & $6,4 \%$ & $10,6 \%$ & $13,2 \%$ & $3,2 \%$ & $0,1 \%$ & $11,0 \%$ & $8,5 \%$ & $100 \%$ \\ 
S2 2023 & $20,3 \%$ & $16,9 \%$ & $11,6 \%$ & $9,4 \%$ & $7,5 \%$ & $13,0 \%$ & $2,8 \%$ & $0,1 \%$ & $10,4 \%$ & $8,1 \%$ & $100 \%$ \\ 
TT & $13,5 \%$ & $15,8 \%$ & $12,0 \%$ & $7,6 \%$ & $7,9 \%$ & $12,5 \%$ & $4,3 \%$ & $0,2 \%$ & $16,5 \%$ & $9,7 \%$ & $100 \%$ \\ 
\end{longtable}

\begin{minipage}{\linewidth}
\textbf{Note de lecture} : 9,1\% des paragraphes de l'ensemble des articles publiés au premier semestre 2020 appartiennent 
à la classe 1 \\ (acheter-petit), c'est-à-dire à la classe constituée des paragraphes qui traitent le sujet du pouvoir d'achat des consommateurs.\\
\textbf{Source} : Corpus inflation-titre réalisé avec Baptiste Yzern.
\end{minipage}

\setlength\LTleft{\holdLTleft}
\setlength\LTright{\holdLTright}

\end{document}
